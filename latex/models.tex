\documentclass[a4paper,12pt]{extreport}

\usepackage{extsizes}
\usepackage{cmap} % для кодировки шрифтов в pdf
\usepackage[T2A]{fontenc}
\usepackage[utf8]{inputenc}
\usepackage[english]{babel}

 \usepackage[usenames, dvipsnames]{color}
\definecolor{fontColor}{RGB}{169, 183, 198}
\definecolor{pageColor}{RGB}{43, 43, 43}

\usepackage{mathtools}

\makeatletter
\let\mytagform@=\tagform@
\def\tagform@#1{\maketag@@@{\color{fontColor}(#1)}}
\makeatother

%\renewcommand\theequation{{\color{fontColor}\arabic{equation_diff}}}

\linespread{1.3} % полуторный интервал
\renewcommand{\rmdefault}{ptm} % Times New Roman
\frenchspacing

\usepackage{graphicx}
\graphicspath{{images/}}
\usepackage{amssymb,amsfonts,amsmath,amsthm}
\usepackage{mathtext}
\usepackage{cite}
\usepackage{enumerate}
\usepackage{float}
\usepackage[pdftex,unicode,colorlinks = true,linkcolor = white]{hyperref}
% \usepackage[pdftex,unicode,colorlinks = true,linkcolor = black]{hyperref}
\usepackage{indentfirst}
\usepackage{placeins}
\bibliographystyle{unsrt}
\usepackage{makecell}
\usepackage{ulem}
\usepackage{longtable}
\usepackage{multirow}
\usepackage{multicol}

\usepackage{tikz}
\usetikzlibrary{arrows,decorations.pathmorphing,
	backgrounds,positioning,fit,petri}

\usepackage{fancyhdr}
\pagestyle{fancy}
\fancyhf{}
\fancyhead[R]{\thepage}
\fancyheadoffset{0mm}
\fancyfootoffset{0mm}
\setlength{\headheight}{17pt}
\renewcommand{\headrulewidth}{0pt}
\renewcommand{\footrulewidth}{0pt}
\fancypagestyle{plain}{
\fancyhf{}
\rhead{\thepage}}

\usepackage{geometry}
\geometry{left=1.5cm}
\geometry{right=1.5cm}
\geometry{top=2.4cm}
\geometry{bottom=2.4cm}

\author{Aleksandr Zhuravlyov}
\title{Numerical model of steady state and transient flow}
\date{\today}


\usepackage {titlesec}
\titleformat{\chapter}{\thispagestyle{myheadings}\centering\hyphenpenalty=10000\normalfont\huge\bfseries}{
\thechapter. }{0pt}{\Huge}
\makeatother


\usepackage{nomencl}
\makenomenclature    % Закомментируйте, если перечень не нужен
%"/usr/texbin/makeindex" %.nlo -s nomencl.ist -o %.nls
\renewcommand{\nomname}{Перечень условных обозначений}
\renewcommand{\nompreamble}{\markboth{}{}}
\newcommand*{\nom}[2]{#1~- #2\nomenclature{#1}{#2}}

\setlength{\columnseprule}{0.4pt}
\setlength{\columnsep}{50pt}
\def\columnseprulecolor{\color{fontColor}}


\begin{document}

    \pagecolor{pageColor}
    \color{fontColor}
    % \maketitle
    % \newpage
    % \tableofcontents{\thispagestyle{empty}}
    % \newpage

    \section*{Single-phase pore network model}

    \begin{eqnarray}
    \label{one_phase} 
    \begin{gathered}
    \sum^{N_{i}}_{j=1, \; i\neq j} \frac{1}{\bar{\rho}_{ij}}G_{ij} = 0,
    \end{gathered}
    \end{eqnarray}
    %
    \begin{eqnarray}
    \begin{gathered}
    \label{press_bound}
    P_{m} = P^{in}   : m \in inlet, \;\;
    P_{m} = P^{out}   :m \in outlet,
    \end{gathered}
    \end{eqnarray}
    %
    \begin{eqnarray}
    \begin{gathered}
    \label{eq:mass_flux_simple_pnm}
    G_{ij} = \bar{\rho}_{ij} \frac{P_{i} - P_{j}}{R_{ij}},
    \end{gathered}
    \end{eqnarray}
    %
     \begin{eqnarray}
    \begin{gathered}
    \bar{\rho}_{ij} = \frac{\rho_{i}+\rho_{j}}{2},
    \end{gathered}
    \end{eqnarray}
    %
    \begin{eqnarray}
        \begin{gathered}
            \alpha_{ij} =\begin{cases}
                             1: &\text{pseudo compressible},\\
                             \frac{P_i + P_j}{2P_i}: &\text{ideal gas},
            \end{cases}
        \end{gathered}
    \end{eqnarray}
    %
    %
\begin{eqnarray}
\begin{gathered}
R_{ij} =\begin{cases}
\frac{8 \mu L_{ij}}{\pi r_{ij}^{4}}: &\text{cylindrical},\\
\frac{12 \mu L_{ij}}{\epsilon_{ij}^{3} \omega_{ij}}: &\text{plates},
\end{cases}
\end{gathered}
\end{eqnarray}
    %
\begin{eqnarray}
\begin{gathered}
G_{in/out} = \sum_{i}^{in/out} \sum^{N_{i}}_{j=1, \; i\neq j} G_{ij}.
\end{gathered}
\end{eqnarray}
    %
\begin{eqnarray}
\begin{gathered}
Q_{in/out} = \frac{G_{in/out}}{\rho_{in/out}}.
\end{gathered}
\end{eqnarray}
    %
\begin{eqnarray}
\begin{gathered}
k = a \frac{Q_{in/out}}{A} \frac{\mu L}{P_{in} - P_{out}},
\end{gathered}
\end{eqnarray}
    %
\begin{eqnarray}
\begin{gathered}
a =\begin{cases}
\frac{\rho_{in/out}}{\bar{\rho}}: &\text{pseudo compressible},\\
\frac{2P_{in/out}}{P_{in} + P_{out}}: &\text{ideal gas},
\end{cases}
\end{gathered}
\end{eqnarray}
    %
\begin{eqnarray}
\begin{gathered}
\bar{\rho}=\frac{1}{P_{in}-P_{out}} \int\limits^{P_{in}}_{P_{out}}\rho \; dP.
\end{gathered}
\end{eqnarray}

    \section*{FVM Diffusion transient flow}

   \begin{eqnarray}
   \label{eq:conductivity_integral}
   \int \limits_{V} \frac{\partial C}{\partial t} d V - \oint \limits_{S} D \vec{\nabla}C d\vec{\Omega} = 0,
   \end{eqnarray}
    \vspace{-0.5cm}
\begin{eqnarray}
\label{eq:conductivity_bound_integral}
C\left(t\right) \Big|_{cleat} = \hat{C} \left(t\right), \;
\vec{\nabla} C \Big|_{outer} = 0,
\end{eqnarray}
    %
    \begin{eqnarray}
    \label{eq:langmuir}
    \hat{C} = \frac{C_L\bar{P}}{P_L + \bar{P}}.
    \end{eqnarray}
    %
    \begin{eqnarray}
    \label{eq:conductivity_num}
    \alpha_i \Delta^{t}_i - \beta_{i-}\Delta^{n+1}_{i-} - \beta_{i+}\Delta^{n+1}_{i+}= 0,
    \end{eqnarray}
    \begin{eqnarray}
    \label{eq:conductivity_bound_num}
    C_1^n = \hat{C}^n, \; \Delta_{M+}^n = 0,
    \end{eqnarray}
    %
\begin{eqnarray}
\label{eq:alpha_beta}
\alpha_i = \frac{\Delta V_i}{\Delta t}, \;
\beta_{i\pm} = \frac{\overline{D}_{i\pm} \Delta \Omega_{i\pm}}{\Delta L_{i\pm}},
\end{eqnarray}
    %
\begin{eqnarray}
\Delta S_{i\pm} =\begin{cases}
h^2: & \text{cartesian},\\
2 \pi h \tilde{r}_{i\pm}: & \text{cylindrical},\\
4 \pi \tilde{r}^2_{i\pm}: & \text{spherical},
\end{cases}
\end{eqnarray}
%
\begin{eqnarray}
V=\begin{cases}
h^2 \left(\tilde{r}_{M+} - \tilde{r}_{1-}\right): & \text{cartesian},\\
\pi h \left(\tilde{r}_{M+}^2 - \tilde{r}_{1-}^2\right): & \text{cylindrical},\\
\frac{4}{3} \pi \left(\tilde{r}_{M+}^3-\tilde{r}_{1-}^3\right): & \text{spherical}.
\end{cases}
\end{eqnarray}
    %
\begin{eqnarray}
\label{eq:delta_num}
\Delta_{i\pm}^n = C_{i\pm1}^n - C_{i}^n, \;
\Delta_i^{t} = C_i^{n+1} - C_i^{n},
\end{eqnarray}
%
\begin{eqnarray}
\label{eq:Consumption_conductivity_integral}
\tilde{G}^{n+1} = -\sum_i^M \alpha_i \Delta_i^{t}.
\end{eqnarray}
\newpage
    \section*{PN and Diffusion coupled}

   \begin{eqnarray}
   \label{eq:main_coupled_sle}
   \begin{gathered}
   \sum^{N_{i}}_{j=1, \; i\neq j} G^{n+1}_{ij} = 0,
   \end{gathered}
   \end{eqnarray}
   %
   \begin{eqnarray}
   \label{eq:mass_flux_coupled}
   \begin{gathered}
   G^{n+1}_{ij} = \bar{\rho}_{ij}^n \frac{P^{n+1}_{i} - P^{n+1}_{j}}{R_{ij}} - \gamma^n_{ij} \tilde{G}^{n+1}_{ij},
   \end{gathered}
   \end{eqnarray}
    %
\begin{eqnarray}
\label{eq:gamma}
\gamma^n_{ij} =\begin{cases}
1: & G_{ij} < 0,\\
0: & G_{ij} \geqslant 0,
\end{cases}
\end{eqnarray}
    %
\begin{eqnarray}
\label{eq:bc_gamma}
\begin{gathered}
\sum^{N_{i}}_{j=1, \; i\neq j} G_{ij} = \hat{G} : i, j \notin outlet,\;\;
P_{m} = P^{out}   : m \in outlet,
\end{gathered}
\end{eqnarray}
%
\begin{eqnarray}
\label{eq:bc_coupled}
\begin{gathered}
\sum^{N_{i}}_{j=1, \; i\neq j} G^{n+1}_{ij} = G^{in}_{i} : i \in inlet,\;\;
P^{n+1}_{m} = P^{out}   : m \in outlet,
\end{gathered}
\end{eqnarray}
%
\begin{eqnarray}
\begin{gathered}
\tilde{G}_{ij}^{n+1} = \tilde{G}_{ij}'^n \cdot \left(\bar{P}_{ij}^{n+1}-\bar{P}_{ij}^n\right)+\tilde{G}_{ij}^n,
\end{gathered}
\end{eqnarray}
%
\begin{eqnarray}
\begin{gathered}
\tilde{G}_{ij}'^n = \frac{\tilde{G}_{ij}\left(\bar{P}_{ij}^n+ \frac{1}{2}\Delta P\right)-
	\tilde{G}_{ij}\left(\bar{P}_{ij}^n-\frac{1}{2}\Delta P\right)}{{\Delta P}},
\end{gathered}
\end{eqnarray}
%
\begin{eqnarray}
\begin{gathered}
\bar{P}^{n}_{ij} = \frac{P^n_i+P^n_j}{2}.
\end{gathered}
\end{eqnarray}
%
\begin{eqnarray}
\begin{gathered}
\tilde{G}_{ij}^{n+1} = \tilde{G}_{ij}^n.
\end{gathered}
\end{eqnarray}

\newpage
\section*{VoF-PNM}

\begin{eqnarray}
\label{one_phase} 
\begin{gathered}
\sum^{N_{i}}_{j=1, \; i\neq j} Q_{ij} = 0,
\end{gathered}
\end{eqnarray}
%
\begin{eqnarray}
\begin{gathered}
\label{press_bound}
P_{m} = P^{in}   : m \in inlet, \;\;
P_{m} = P^{out}   :m \in outlet,
\end{gathered}
\end{eqnarray}
%
 \begin{eqnarray}
\begin{gathered}
\label{eq:twophase_mass_flux}
Q_{ij} = \frac{P_{i} - P_{j} + \Delta P_{ij}}{R_{ij}},
\end{gathered}
\end{eqnarray}
%
 \begin{eqnarray}
\begin{gathered}
\label{eq:twophase_mass_hydr_conductance}
R = \frac{12 \bar{\mu} L}{h^{3} \omega},
\end{gathered}
\end{eqnarray}
%
\begin{eqnarray}
\begin{gathered}
Q_{in/out} = \frac{G_{in/out}}{\rho_{in/out}}.
\end{gathered}
\end{eqnarray}
%
\begin{eqnarray}
\begin{gathered}
k = a \frac{Q_{in/out}}{A} \frac{\mu L}{P_{in} - P_{out}},
\end{gathered}
\end{eqnarray}
%
\begin{eqnarray}
\begin{gathered}
a =\begin{cases}
\frac{\rho_{in/out}}{\bar{\rho}}: &\text{pseudo compressible},\\
\frac{2P_{in/out}}{P_{in} + P_{out}}: &\text{ideal gas},
\end{cases}
\end{gathered}
\end{eqnarray}
%
\begin{eqnarray}
\begin{gathered}
\bar{\rho}=\frac{1}{P_{in}-P_{out}} \int\limits^{P_{in}}_{P_{out}}\rho \; dP.
\end{gathered}
\end{eqnarray}

\newpage
 \section*{Two-phase pore network model (coupling of  PNM and DFS)}
 
 \subsection*{Model description:} \label{s1}
 \begin{enumerate}
 	\item Set initial and boundary conditions, in terms of saturation, to the model. Having that, use Eqs. (\ref{eq:twophase_av_saturation_grad}) and (\ref{eq:twophase_dens_visc}) to calculate average saturation gradient, density and viscosity inside each of the throats. 
 	\item Calculate the capillary pressure ($P^c$). Estimate the corresponding pressure drop ($\Delta P$) inside the fractures which has a functional dependence on $P^c$ and saturation gradient ($\bar{\nabla} S$).
 	\item Calculate the average velocity inside the throats by solving Eq. (\ref{eq:twophase_mass_flux}).
 	\item Solve Eqs. (\ref{eq:twophase_integral_w}) and (\ref{eq:twophase_gas_phase}) for a corresponding time step to obtain a new saturation profile.
 \end{enumerate}
 
 The proposed model represents the modified volume of fluid (VOF) approach used in conjunction with PNM to predict two-phase flow in fractured media.  
 Here we modify Eq. (\ref{eq:mass_flux_simple_pnm}) to account for pressure drop due to capillary forces during two-phase flow. 
 \begin{eqnarray}
 \begin{gathered}
 \label{eq:twophase_mass_flux}
 G_{ij} = \bar{\rho}_{ij} \frac{P_{i} - P_{j} + \Delta P_{ij}}{R_{ij}},
 \end{gathered}
 \end{eqnarray}
  where $\Delta P$ represents the additional pressure gradient due to capillary between two fluids inside the cuboid fracture.
  
  Other equations are written for an arbitrary fracture. Thus, the coupled index $ij$, representing an arbitrary fracture, is omitted for simplicity.
  
  \begin{eqnarray}
  \begin{gathered}
  \label{eq:twophase_mass_hydr_conductance}
  R = \frac{12 \bar{\mu} L}{h^{3} \omega},
  \end{gathered}
  \end{eqnarray}
  where $h$~--~fracture height, and $\omega$~--~fracture width.
  
  The continuity equations for phases flowing through a system of fractures can be expressed as follows:
  \begin{eqnarray}
  \label{eq:twophase_integral_w}
  \int \limits_{V} \frac{\partial S_0}{\partial t} d V - \oint \limits_{\Omega} \vec{v}S_0 d\vec{\Omega} = 0,
  \end{eqnarray}
  
  \begin{eqnarray}
  \label{eq:twophase_gas_phase}
  S_1 = 1 - S_0,
  \end{eqnarray}
  where  $S$ is saturation, $0$~--~first phase, $1$~--~second phase, $\vec{v}$ is velocity of fluid mixture inside the throat. We are omitting $0$ index for simplicity in further equations.
 \newpage
The following initial (\ref{eq:twophase_saturation_init}) and boundary conditions (\ref{eq:twophase_saturation_boundary}) have been applied:
  
  \begin{eqnarray}
  \label{eq:twophase_saturation_init}
  S \left(\vec{x}, \mathit{0}\right) = \hat{S} \left(\vec{x}\right),
  \end{eqnarray}
  
  \begin{eqnarray}
  \label{eq:twophase_saturation_boundary}
  S \Big|_{\mathit{\Gamma}_D} \!\!= \tilde{S}, \; \vec{v}S \Big|_{\mathit{\Gamma}_N} \!\!= \vec{v}\Big|_{\mathit{\Gamma}_N}S \Big|^{vic}_{\mathit{\Gamma}_N},
  \end{eqnarray}
  
  The finite-volume representation of Eqs. (\ref{eq:twophase_integral_w}), (\ref{eq:twophase_saturation_init}), and (\ref{eq:twophase_saturation_boundary}) can be described as follows:
  
   \begin{eqnarray}
  \label{eq:twophase_num}
  \alpha S^{n+\mathit{1}} -  \sum_{\Delta S} \beta^n\bar{S}^{n+1} =   \alpha S^{n},
  \end{eqnarray}
  
  \begin{eqnarray}
  \label{eq:twophase_num_init}
  S = \hat{S}^{\mathit{0}}, \; 
  \end{eqnarray}
  \begin{eqnarray}
  \label{eq:twophase_num_bound}
  S^{n+\mathit 1}  \Big|_{\mathit{\Gamma}_D}= \tilde{S}^{n+\mathit 1}, \; \beta^n \bar{S}^{n+\mathit 1} \Big|_{\mathit{\Gamma}_N} \!\!= \beta^n S^{n+\mathit 1} \Big|_{\mathit{\Gamma}_N},
  \end{eqnarray}
  where parameters $\alpha$ and $\beta$ are expressed as follows:
  
  \begin{eqnarray}
  \label{eq:twophase_alpha_beta}
  \alpha = \frac{\Delta V}{\Delta t}, \;
  \beta^n = v^n\Delta \Omega,
  \end{eqnarray}
  where $\Delta V$~--~volume of a particular grid-block, $\Delta t$~--~numerical time step, $\Delta L$~--~distance between centres of two neighbouring grid-blocks, $\Delta \Omega$~--~surface bounding a grid-block (positive or negative sign is defined by the normal direction).

\begin{eqnarray}
\label{eq:twophase_delta_num}
\bar{S}^{n+\mathit 1} =\frac{\sum_{\Delta S}S_{upw}^{n+\mathit{1}}\beta^n_{upw}}{\sum_{\Delta S}\beta^n_{upw}},
\end{eqnarray}
 where index $upw$ denotes the upwind discretization method.
 
 The capillary pressure ($P^c$)  and the corresponding pressure drop ($\Delta P$) inside the fracture can be calculated using the following equations:
  \begin{eqnarray}
  \begin{gathered}
  \label{eq:twophase_capillary_pressure_fractures}
  P^{c} = 2 \gamma cos \theta \left[\frac{1}{h} + \frac{1}{w}\right], \; \Delta P = \Delta P \left(\tilde{\nabla} S, P^c\right),
  \end{gathered}
  \end{eqnarray}
  where $\gamma$~--~interfacial tension, $\theta$~--~contact angle, $h$ is a fracture height and $w$ is a fracture width.
  \begin{eqnarray}
  \label{eq:twophase_av_saturation_grad}
  \begin{gathered}
  \tilde{\nabla} S = \int_{0}^{L} \nabla S dL \simeq\sum_{k=1}^{M} \nabla S_{k} \Delta L_{k},
  \end{gathered}
  \end{eqnarray}
 where $\tilde{\nabla} S$ is a phase saturation gradient inside the fracture, $M$ is a total number of grid blocks, $\Delta L$~--~grid block length, $k$~--~grid block index.
 
 
Average density and viscosity inside the fracture can be calculated as follows:
  \begin{eqnarray}
  \label{eq:twophase_dens_visc}
  \begin{gathered}
  \bar{\rho}^i = \frac{1}{M^i}\sum_{j=1}^{M^i} \left(S_0^{ij} \rho^{ij}_0+ \left(1-S_0^{ij}\right) \rho^{ij}_1\right), \\
  \bar{\mu}^i = \frac{1}{M^i}\sum_{j=1}^{M^i} \left(S_0^{ij} \mu^{ij}_0+ \left(1-S_0^{ij}\right) \mu^{ij}_1\right),
  \end{gathered}
  \end{eqnarray}
  
Relative permeability estimation:
\begin{eqnarray}
	\label{eq:single_0_velocity}
	\begin{gathered}
		\phi \bar{v}_0^*  = -\frac{k}{\mu_0} \frac{\Delta P_0^*}{\Delta L}
	\end{gathered}
\end{eqnarray}
\begin{eqnarray}
	\label{eq:single_1_velocity}
	\begin{gathered}
		\phi \bar{v}_1^*  = -\frac{k}{\mu_1} \frac{\Delta P_1^*}{\Delta L}
	\end{gathered}
\end{eqnarray}

\begin{eqnarray}
	\label{eq:rel_0}
	\begin{gathered}
		k_{r0}=F\frac{\bar{v}}{\bar{v}_0^*} \frac{\Delta P_{0}^*}{\Delta P_0}
	\end{gathered}
\end{eqnarray}
\begin{eqnarray}
	\label{eq:rel_1}
	\begin{gathered}
		k_{r1}=\left(\mathit{1}-F\right)\frac{\bar{v}}{\bar{v}_\mathit{1}^*} \frac{\Delta P_{\mathit{1}}^*}{\Delta P_\mathit{1}}
	\end{gathered}
\end{eqnarray}
\begin{eqnarray}
	\label{eq:sat_av}
	\begin{gathered}
		\bar{S}_0 = \frac{\sum_{i=1}^N S_{0}^i V^i}{\sum_{i=1}^N V^i}, \\
	\end{gathered}
\end{eqnarray}
\begin{eqnarray}
	\label{eq:bl_func}
	\begin{gathered}
		F = \frac{\sum_{i=\mathit{1}}^N S_{0}^i v^i V^i}{\sum_{i=\mathit{1}}^N v^iV^i}, \\
	\end{gathered}
\end{eqnarray}
\begin{eqnarray}
	\label{eq:twophase_dens_visc_model}
	\begin{gathered}
		\bar{\rho} = \bar{S}_{0} \rho_0+ \left(1-\bar{S}_0\right) \rho_1, \\
		\bar{\mu} = \bar{S}_{0} \mu_0+ \left(1-\bar{S}_0\right) \mu_1. \\
	\end{gathered}
\end{eqnarray}

\end{document}
